\PassOptionsToPackage{unicode=true}{hyperref} % options for packages loaded elsewhere
\PassOptionsToPackage{hyphens}{url}
%
\documentclass[]{article}
\usepackage{lmodern}
\usepackage{amssymb,amsmath}
\usepackage{ifxetex,ifluatex}
\usepackage{fixltx2e} % provides \textsubscript
\ifnum 0\ifxetex 1\fi\ifluatex 1\fi=0 % if pdftex
  \usepackage[T1]{fontenc}
  \usepackage[utf8]{inputenc}
  \usepackage{textcomp} % provides euro and other symbols
\else % if luatex or xelatex
  \usepackage{unicode-math}
  \defaultfontfeatures{Ligatures=TeX,Scale=MatchLowercase}
\fi
% use upquote if available, for straight quotes in verbatim environments
\IfFileExists{upquote.sty}{\usepackage{upquote}}{}
% use microtype if available
\IfFileExists{microtype.sty}{%
\usepackage[]{microtype}
\UseMicrotypeSet[protrusion]{basicmath} % disable protrusion for tt fonts
}{}
\IfFileExists{parskip.sty}{%
\usepackage{parskip}
}{% else
\setlength{\parindent}{0pt}
\setlength{\parskip}{6pt plus 2pt minus 1pt}
}
\usepackage{hyperref}
\hypersetup{
            pdftitle={12. Nullstellensatz},
            pdfborder={0 0 0},
            breaklinks=true}
\urlstyle{same}  % don't use monospace font for urls
\setlength{\emergencystretch}{3em}  % prevent overfull lines
\providecommand{\tightlist}{%
  \setlength{\itemsep}{0pt}\setlength{\parskip}{0pt}}
\setcounter{secnumdepth}{0}
% Redefines (sub)paragraphs to behave more like sections
\ifx\paragraph\undefined\else
\let\oldparagraph\paragraph
\renewcommand{\paragraph}[1]{\oldparagraph{#1}\mbox{}}
\fi
\ifx\subparagraph\undefined\else
\let\oldsubparagraph\subparagraph
\renewcommand{\subparagraph}[1]{\oldsubparagraph{#1}\mbox{}}
\fi

% set default figure placement to htbp
\makeatletter
\def\fps@figure{htbp}
\makeatother


\title{12. Nullstellensatz}
\date{}

\begin{document}
\maketitle

\hypertarget{hilberts-nullstellensatz}{%
\subsection{Hilbert's Nullstellensatz}\label{hilberts-nullstellensatz}}

\hypertarget{radicals-and-radical-ideals}{%
\subsubsection{Radicals and Radical
Ideals}\label{radicals-and-radical-ideals}}

\textbf{Definition:} If \(I\subset R\) is an ideal, \(I\) is called
\emph{radical} if, whenever \(f^n\in I\), we have \(f\in I\).\\
Alternatively, \(I\) is radical if \(R/I\) has no nilpotent elements. If
\(I\) is any ideal, then \(\mathrm{rad}(I)\) is the set of elements
\(f\) such that \(f^{m}\in I\) for some \(m\). Finally, the radical of
the zero ideal, which is the set of nilpotent elements in \(R\), is
called the \emph{nilradical} of \(R\).

\textbf{Remark:} We've seen at various times in the past that the
nilpotent elements of a (commutative) ring form an ideal.

\textbf{Proposition:} If \(I\) is a proper ideal of \(R\), then the
radical of \(I\) is the intersection of all the prime ideals of \(R\)
containing \(I\).

\textbf{Proof:} It's enough to prove that the nilradical of \(R/I\) is
the intersection of all prime ideals of \(R/I\). If \(P\supset I\) is a
prime ideal, and \(f^{n}\in I\) for some \(n\), choose the smallest such
\(n\). Then \(f^{n}\in P\) so either \(f^{n-1}\in P\) or \(f\in P\). By
minimality of \(n\), this means that \(f\in P\). So the nilradical is
contained in every prime ideal.

For the converse, suppose that \(a\) is not a nilpotent element of \(R\)
(and is not a unit in \(R\)). Then we will construct a prime ideal \(P\)
that does not contain \(a\). Let \(A\) be the set of powers of \(a\):
\(A=\{a,a^2,a^3,\ldots\}\) and let \(S\) be the set of ideals of \(R\)
not meeting \(A\). This is a nonempty set, since it contains the zero
ideal. If \(I_1\subset I_2\subset\cdots\) is a chain of ideals in \(S\),
then the union of the \(I_{k}\) is again an ideal in \(S\), so chains in
\(S\) have upper bounds. By Zorn's lemma, \(S\) has a maximal element
\(Q\). Now suppose that \(x\) and \(y\) are elements of \(R\) and
\(xy\in P\). Since \(P\) is maximal in \(S\), we know that some power of
\(a^r\) is in \((x)+P\) and some power of \(a^s\) is in \((y)+P\). But
then \(a^{r+s}\) is in \(xy+P=P\) since \(xy\in P\). This is a
contradiction, since \(P\) is in \(S\). It follows that one of \(x\) or
\(y\) must have been in \(P\), so \(P\) is prime.

\textbf{Corollary:} Prime (and maximal) ideals of \(R\) are radical
ideals.

\hypertarget{integral-extensions}{%
\subsubsection{Integral Extensions}\label{integral-extensions}}

\textbf{Definition:} Let \(S\) be a commutative \(R\) algebra.

\begin{itemize}
\tightlist
\item
  An element \(a\in S\) is integral over \(R\) if it is the root of a
  monic polynomial in \(R[x]\).
\item
  If every element of \(s\) is integral over \(R\), then \(S\) is called
  an \emph{integral} extension of \(R\).
\item
  The subset of \(S\) consisting of elements integral over \(R\) is
  called the \emph{integral closure} of \(R\) in \(S\).
\item
  \(R\) is integrally closed in \(S\) if it is equal to its integral
  closure.
\item
  If \(R\) is an integral domain, and \(R\) is integrally closed in its
  field of fractions, then \(R\) is integrally closed (full stop) or
  \emph{normal}. The integral closure of \(R\) in its field of fractions
  is called its normalization.
\end{itemize}

\textbf{Proposition:} The following are equivalent:

\begin{itemize}
\tightlist
\item
  \(a\) is integral over \(R\).
\item
  \(R[a]\) is a finitely generated \(R\) module.
\item
  There is a subring \(R\subset T\subset S\) containing \(a\) wuch that
  \(T\) is a finitely generated \(R\)-module
\end{itemize}

\textbf{Proof:} If \(a\) satisfies the monic polynomial
\(x^n+r_{n-1}x^{n-1}+\cdots+r_0\), then any element of \(R[a]\) can be
written as a linear combination of \(1,a,a^2,\ldots, a^{n-1}\). So
\(R[a]\) is finitely generated. The ring \(R[a]\subset S\) is a finitely
generated \(R\) module inside \(S\). Finally, if \(a\) belongs to a
finitely generated \(R\) module \(T\), choose generators for \(T\)
\(t_1,\ldots, t_n\) over \(R\) and consider the equations

\[
at_{i}=\sum r_{ij}t_{j}
\]

The element \(a\) satisfies the (monic) characteristic polynomial made
from the entries \(r_{ij}\), so \(a\) is integral over \(R\).

\textbf{Corollary:} The sum and product of integral elements are
integral; the integral closure of \(R\) in \(S\) is a subring of \(S\);
and if \(S\) is integral over \(R\) and \(T\) is integral over \(S\)
then \(T\) is integral over \(R\).

\textbf{Corollary:} Let \(\tilde{R}\) be the integral closure of \(R\)
in \(S\). Then \(\tilde{R}\) is integrally closed.

\textbf{Proof:} If \(x\in S\) is integral over \(\tilde{R}\), then since
\(\tilde{R}\) is integral over \(R\), we have \(x\) is integral over
\(R\) so belongs to \(\tilde{R}\).

\textbf{Proposition:} Suppose that \(S\) is an \(R\)-algebra that is
integral over \(R\). Then \(R\) is a field if and only if \(S\) is a
field.

\textbf{Proof:} Suppose first that \(R\) is a field. Choose \(s\in S\).
Then

\[
s^{n}+r_{n-1}s^{n-1}+\cdots+r_{0}=0
\]

where we can assume \(r_{0}\not=0\). Then

\[
s(s^{n-1}+r_{n-1}s^{n-2}+\cdots+r_{1})=-r_{0}.
\]

Since \(-r_0\not=0\), we can divide the polynomial on the right by
\(-r_{0}\) to obtain a multiplicative inverse for \(s\).

Now suppose that \(S\) is a field. If \(r\in R\), then \(r\in S\), so
\(r^{-1}\in S\). We have

\[
r^{-m}+r_{m-1}r^{-m-1}+\cdots+r_{0}=0
\]

so by clearing demoninators we can write \(r^{-1}\) as an element of
\(R\).

\hypertarget{noether-normalization}{%
\subsubsection{Noether Normalization}\label{noether-normalization}}

\textbf{Definition:} Elements \(x_1,\ldots, x_n\) in a \(k\)-algebra
\(S\) are called algebraically independent if there are no nonzero
polynomial relations among them: there are no polynomials \(p\) so that
\(p(x_1,\ldots, x_n)=0\). In other words, they generate a copy of
\(k[x_1,\ldots, x_n]\subset S\).

\textbf{Theorem:} (Noether Normalization) Let \(k\) be a field and let
\(A\) be a finitely generated \(k\)-algebra. Then there are
algebraically independent elements \(y_1,\ldots, y_q\) in \(A\) such
that \(A\) is integral over \(k[y_1,\ldots, y_q]\).

\textbf{Proof:} The proof is by induction and is (more or less)
algorithmic. Start with generators \(x_1,\ldots, x_n\) for \(A\). If
they are algebraically independent, you're done. Otherwise you have a
polynomial relation

\[
p(x_1,\ldots, x_n)=0.
\]

This is a sum of monomials \(x_1^{a_1}\cdots x_n^{a_n}\). The degree of
\(p\) is the largest of the sums of these exponents; call that \(d\).
Then let \(\alpha\) be any integer bigger than \(d\) (\(d+1\) works
fine).

Introduce new coordinates \(X_i\) (for \(i=1,\ldots, n-1\)) by:

\[
\begin{array}{rcl}
x_1 &=& X_1 + x_n^{\alpha}\\
x_2 &=& X_2 + x_n^{\alpha^2}\\
\vdots &=& \vdots\\
x_{n-1} &=& X_{n-1} + x_n^{\alpha^{n-1}}\\
\end{array}
\]

If we substitute the new coordinates, we get
\(p(X_1+x_m^{\alpha},\cdots, X_{n-1}+x_n^{\alpha^{n-1}},x_n)=0\). But a
monomial \(x_1^{a_1}\cdots x_n^{a_n}\) will contribute a term

\[
x_n^{a_n+a_{n-1}\alpha^{n-1}+a_{n-2}\alpha^{n-2}+\cdots+a_1\alpha}
\]

and since we choose \(\alpha\) bigger than \(d\) we have all
\(a_i<\alpha\). In other words, all of these exponents of \(x_n\) are
distinct (they are different in base \(\alpha\)).

It follows that the polynomial
\(p(X_1+x_m^{\alpha},\cdots, X_{n-1}+x_n^{\alpha^{n-1}},x_n)\) has the
form

\[
p(X_1+x_m^{\alpha},\cdots, X_{n-1}+x_n^{\alpha^{n-1}},x_n)=cx_m^{N}+\sum H_{i}(X_1,\ldots, X_{n-1})x_m^{i}
\]

and so \(x_m\) is integral over the subring
\(B=k[X_1,\ldots, X_{m-1}]\). But then \(x_i\) for \(i=1,\ldots,n-1\)
are integral over \(B[x_{m}]\) because they satisfy the equations
\(x_i-X_{i}-x_{n}^{\alpha^{i}}\). Therefore \(A\) is integral over \(B\)
(which has fewer generators). Continue by induction.

\textbf{Theorem:} (the ``weak'' nullstellensatz) Let \(k\) be an
algebraically closed field and let \(A=k[x_1,\ldots, x_n]\). Then the
maximal ideals \(M\) of \(A\) are all of the form
\[M=(x-a_1,\ldots, x-a_n)\] where the \(a_i\in k\).

\textbf{Corollary:} The correspondence between ideals and algebraic sets
gives a bijection between points and maximal ideals of
\(\mathbb{A}^{n}_{k}\).

\textbf{Corollary:} Let \(f_1,\ldots, f_k\in A\). Then either the
\(f_{i}\) have a common zero, or there are polynomials
\(g_1,\ldots, g_k\) in \(A\) such that

\[
1=\sum g_{i}f_{i}.
\]

\textbf{Proof:} (of the theorem) Clearly an ideal of the form
\((x_1-a_1,\ldots, x_n-a_n)\) is maximal, so suppose \(M\) is a maximal
ideal of \(A\). Let \(E=A/M\). Then \(E\) is a finitely generated
\(k\)-algebra, so there are algebraically independent elements
\(y_1,\ldots, y_k\) such that \(E\) is integral over
\(k[y_1,\ldots, y_k]\). But \(E\) is a field, so \(k[y_1,\ldots, y_k]\)
is a field. But this can only happen if \(k=0\). Then \(E/k\) is a
finite integral (i.e.~algebraic) extension of \(k\), and \(k\) is
algebraically closed, so \(E=k\). This means that each of the generators
\(x_i\) is congruent mod \(M\) to an element of \(k\), or in other words
\(M\) is of the desired form.

For the corollaries, any proper ideal \(I\) of \(A\) is contained in a
maximal ideal \(M\), so if \(X(I)\) contains the point corresponding to
\(M\). So the points of \(X(I)\) correspond to the maximal ideals
containing \(I\).

Finally, if the \(f_i\) have no common zero, then they must not generate
a proper ideal, so the ideal they generate contains \(1\).

\textbf{Theorem:} (Nullstellensatz, ``strong'' form) Let \(k\) be an
algebraically closed field. Then if \(J\subset A\) is any ideal,
\(I(X(J))=\mathrm{rad}(J)\). Thus (assuming \(k\) is algebraically
closed) there are mutually inverse bijections between algebraic sets in
\(\mathbb{A}^{n}_{k}\) and radical ideals in \(A\).

\textbf{Proof:} We know that \(\mathrm{rad}(J)\subset I(X(J))\) so we
need to prove the opposite. We know that \(J\) is finitely generated,
say by \(f_1,\ldots, f_k\). Let \(g\) be any polynomial vanishing on
\(X(J)\). Make a new ring \(A'\) by introducing a new variable
\(x_{n+1}\) and a new ideal \(J'\in A'\) by adding the relation
\(gx_{n+1}-1\). (Notice that this means that \(x_{n+1}\) is \(1/g\).) If
the elements of \(J'\) had a common zero, all of the \(f_{i}\) would
vanish at that point and since \(g\in I(X(J))\) so would \(g\). But that
doesn't happen, so \(J'\) can't be a proper ideal and so we have an
equation

\[
1=h_1 f_1 +\cdots + h_k f_k + h_{k+1}(x_{n+1}g-1)
\]

We can divide this equation by a high power of \(x_{n+1}\) so that the
powers of \(h_{n+1}\) in the coefficients \(h_{i}\) are all negative. In
other words, writing \(y=1/x_{n+1}\), we get an equation

\[
y^{N} = b_1 f_1 +\cdots +b_k f_k + b_{k+1}(g-y)
\]

where the \(b_{i}\) are polynomials in \(x_1,\ldots, x_n\) and \(y\).
This is an equation in \(A'\), where \(g=1/x_{n+1}\), so this means
(substituting \(g\) for \(y\)) that we have an equation showing that
\(g^{N}\) is in the ideal generated by the \(f_{i}\), so
\(g\in\mathrm{rad}(J)\).

\textbf{Corollary:} If \(k\) is not algebraically closed, then we can
still conclude that a set of polynomials that generates a proper ideal
of \(k[x_1,\ldots, x_n]\) must have common zeros in the algebraic
closure of \(k\).

\end{document}
